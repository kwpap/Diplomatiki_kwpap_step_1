\begin{theorem}

\end{theorem}

\begin{proof}
Let $\vec{x}$ be any $5-$stable instance with optimal clustering $\vec{C} = (C_1,...,C_k)$. Let $\vec{x}' = (\vec{x}_{-i},x')$ be the resulting instance after agent $i$ deviates from location $x_i$ to $x'$, with optimal clustering $\vec{C'}$. In the original clustering $x_i\in C_i$. Since the mechanism places a facility at a location of an agent selected uniformly at random from each optimal cluster, no agent can gain by deviating within the range of her optimal cluster. Therefore we need to cover all $i$'s possible deviations that result in a different clustering than the original. This can happen in three ways:

\begin{itemize}
\setlength\itemsep{0.1em}
  \item[]\textbf{Case 1:} Becoming a member of another cluster
  \item[]\textbf{Case 2:} Becoming a self serving cluster
  \item[]\textbf{Case 3:} Merging or splitting $C_i$
\end{itemize}


In case 1, agent $i$ is clustered together with members from a different cluster in $\vec{C}$ without splitting or merging $C_i$. The deviation from $x_i$ to $x'$ results in either an increase in her expected cost or a violation of the cluster separation property in $\vec{C}'$.


First, we need to compute the expected $i$'s cost if she reports her true location. We can define as $X_i$ the discrete random variable that takes values from the sample space $\{ d(x,x_i): x\in C_i\}$ uniformly at random. Let $|C_i|=n$. Then, her expected cost is:

\[ E[X_i] = \frac{\sum_{x\in C_i} d(x,x_i)}{n} \]

Note that $i$'s cost is at most $D(C_i)$ since the center is selected uniformly at random among the agents in $C_i$.



In this case, agents in $C_i$ are not merged or splitted in $\vec{C}'$. So, all the agents that were originally cluster together with $x_i$ in $\vec{C}$ are cluster together in $\vec{C}'$. Let that be $C_i'$. %($x\in C_i\setminus x_i$) 
%needs haaalp!!
Let $C_j'$ be the cluster that $x_i$ belongs in $\vec{C}'$. Since she is trying to make the deviation profitable $C_j'$ need to be the closest cluster, different from $C_i'$, to her original location.


After this change in the clustering we need to compute $i$
's expected cost. Same as before, we can define as $X_i'$ the discrete random variable that takes values from the sample space $\{ d(x,x_i): x\in C_i' \}  \equiv \{ d(x,x_i): x\in C_i, x\neq x_i \}$ and $X_j'$ the discrete random variable that takes values from the sample space $\{d(x',x_i)\} \	\cup \{ d(x,x_i): x\in C_j'\}$ uniformly at random. $X_i'$ represents $i$'s cost if she is served by the facility placed in $C_i'$ and $X_j'$ her cost if she is served by the facility in $C_j'$. Her expected cost becomes $\mathbb{E} [min\{ X_i', X_j' \}]$, since she can choose the facility closest to $x_i$. 
By the cluster separation property for any agent $x_j \notin C_i$ it holds that $d(x_i,x_j)>D(C_i)$.  We know that: 
\[\mathbb{E}[X_j'] =  \frac{\sum_{x \in C_j'} d(x,x_i) + d(x',x_i)}{|C_j'|+1}\]

If $d(x_i,x') > D(C_i)$ then all the values from $X_j'$'s sample space are larger than  $X_i$'s. Thus, agent i will always has small expected cost if she chooses the facility placed in $C_i'$. 
\[\mathbb{E}[min\{X_i',X_j'\}] = \mathbb{E}[X_i'] = \frac{\sum_{x \in C_i'} d(x,x_i)}{n-1} = \frac{\sum_{x \neq x_i\in C_i} d(x,x_i)}{n-1} > \mathbb{E}[X_i]\]

Since she is not a member of $C_i'$ she cannot affect the facility's location, making her expected cost higher than her original expected cost.
\bigskip

The only way she can gain by the deviation is if $x'$ belongs in $C_j$ and $d(x',x_i)<D(C_i)$. This way, she is close enough to her original location to make the deviation profitable but far enough to change the original clustering. Consider the case where $c_i$, $x_i$ and $x_j$ are collinear, $x_j$ being the agent from $C_j$ closest to $x_i$. In any other case, the distance $d(x',C_i')$ will be smaller, making the deviation easier to detect. 

By the cluster separation property for the stability factor of $5$, we have that  $d(x_i,x_j) = d(C_i,C_j) \ge 1.6\cdot D(C_i)$. In order for this distance to be tight all it must be $d(x_i,c_i) = 0.4D(C_i)$. We also have that $d(x_j,c_j) < 0.4D(C_i)$, since by stability it holds $d(x_i,x_j) > (\gamma-1)d(x_j,c_j)$. Combining the above inequalities and that $d(x',x_i)<D(C_i)$ we get:
\begin{align}
d(x',c_i) \le d(x',x_i) + d(x_i,c_i) \le 1.4D(C_i) \\
d(x',x_j)>d(x_i,x_j) - d(x_i,x') > 0.6D(C_i)
\end{align}


Finally we need to distinguish two cases regarding the cluster $C_j'$: 
\begin{enumerate}
    \item $c_j \in C_j'$:  We have that $D(C_j') \ge d(c_j,x')$ and that $d(C_i',C_j') \le d(c_i,x') \le 1.4 D(C_i)$. By stability it hold $d(x_i,c_j) > \gamma d(x_i,c_i) = 2D(C_i)$. From triangle inequality it holds $d(c_j,x') \ge  d(c_j,x_i)-d(x_i,x') > D(C_i)$. Therefore, we get that $d(C_i',C_j') \le 1.4 D(C_i) < 1.4 D(C_i')$ 
    
    \item $c_j \notin C_j'$: In this case the agents from $C_j$ are splitted in (at least) two clusters in $\vec{C}'$. Let $C_l'$ be the cluster than $c_j$ belongs to. Since $x_j\in C_j'$ and $c_j \in C_l'$ we have that $d(C_j',C_l') \le d(c_j,x_j) < 0.4D(C_i)$. But $D(C_j') \ge d(x_j,x')>0.6D(C_i)$. Combining this inequalities we get $D(C_j',C_l') \le 0.67 D(C_j')$ which violates the minimum distance between  $C_j'$ \& $C_l'$.
\end{enumerate}

\end{proof}


