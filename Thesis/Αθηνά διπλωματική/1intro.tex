``How can a group of individuals choose a winning outcome from a given set of options?" This is a question that has been studied in various fields like sociology (social choice theory), economics (game theory), and most recently, computer science (algorithmic game theory and computation social choice). Throughout this thesis, we will focus on the algorithmic aspect of the previous question and, specifically, on how we can effectively achieve a desirable outcome. These types of problems are studied in the field of \textit{Algorithmic Mechanism Design}, an intersection between \textit{Economic Theory} and \textit{Computer Science}. In classic mechanism design, the goal is to design a system for multi-participant environments. Depending on the setting, we are interested in different performance objectives like revenue maximization and social welfare maximization. Since the agents are strategic and rational, the system needs to ensure that all the agents will behave as the designer intended. Algorithmic mechanism design also considers computational constraints. Using tools from theoretical computer science while respecting game-theoretic constraints, the goal is to design efficient mechanisms.


%The prototypical problem in mechanism design is to design a system for multiple self-interested participants, such that the participants' self-interested actions at equilibrium lead to good system performance. Typical objectives studied


%Algorithmic game theory, as the name suggests, is an area in the intersection of game theory and computer science, with the objective of understanding and design of algorithms in strategic environments. We can see Algorithmic Game Theory from two perspectives: \textbf{\textit{Analysis:}} given a game with a predefined set of rules the goal is to analyze the outcome based on the behavior of the strategic agents (e.g., calculate and prove properties on their Nash equilibrium or compute the price of anarchy) \textbf{\textit{Design:}} design games that have both good game-theoretical properties, such as truthfulness and socially desirable outcome, and algorithmic properties, namely computational efficiency. This area is called ``\textit{algorithmic mechanism design.}"


\section{Motivation}

The \emph{Facility Location Problem} is one of the most fundamental and well-studied problems in Theoretical Computer Science, Operation Research, and recently Algorithmic Mechanism Design. Such problems are motivated by natural scenarios in Social Choice, where the government plans to build a fixed number of public facilities in an area like schools, libraries, or hospitals(e.g., see \cite{Miyagawa}). Apart from locating actual facilities, the problem has a wide range of applications, including non-geographic problems such as selecting a committee to represent people with differing political views. 


Let us look at it as an optimization problem first. The goal is to find the optimal placement of facilities to minimize transportation costs given a set of locations in a metric space. However, there are many applications where the locations are not publicly known and have to be reported by strategic agents. Now the goal is to design a strategyproof mechanism, i.e., does not incentivize agents to lie, which is also efficient with respect to the optimal solution. In many settings in Mechanism Design (e.g., auctions), payments guarantee that the optimal solution is strategyproof. On the other hand, payments could be illegal or unethical in Social Choice environments, such as Facility Location games\cite{SV07}. Procaccia and Tennenholtz \cite{Procaccia2013} showed that strategyproofness could also be achieved without payments by sacrificing the solution's optimality, thus initiating the research on \emph{Approximate Mechanism design without Money}.

Since then, the problem has been studied in many different settings and generalizations. There are two survey papers to get a complete overview of the problem: one for mechanism design for Facility Location Problems \cite{Chan2021} and one for approximation algorithms for Facility Location and Clustering Problems \cite{An2017}. In this thesis, we are going to focus on the mechanism design aspect of the problem. This problem has been studied for different metric spaces. One of the most researched setting is the $k$-Facility Location on the line (\cite{Fotakis2014,Fotakis2013sp,GolombT17,Lu2009,Nissim2010}).  It is also been researched for restricted metric spaces more general that the line(e.g., trees, circles, and plane \cite{Alon2010,Dokow2012,Filimonov2021,Goel2020,Meir2019}) and general metric spaces (\cite{Fotakis2013, Lu2010}). Another direction is to design mechanism for different objective functions, like \emph{maximum cost} \cite{Procaccia2013,Fotakis2013sp} and \emph{mini-sum-of-squares}\cite{Feldman2011}. For all those variations we assume that all the facilities are homogeneous. Recently, there is some work where the facilities serve different purposes (\cite{Li2021,Kyropoulou2019,Serafino2016}). There are also considered not single-peaked preference profiles   (\cite{MeiLYZ19,CHEN2020185,Feigenbaum2020})



In this thesis, we are going to focus on the classic $k$-Facility Location games, where $k$ uncapacitated facilities are placed in a metric space based on the preferences of $n$ strategic agents. Our goal is to minimize the social cost objective, namely the sum of the distances from each agent to the nearest facility. When the agents are located on the real line, we have a complete characterization of deterministic strategyproof mechanisms. For one facility, the mechanism that places the facility at the median location is strategyproof and optimal with respect to the social cost \cite{Procaccia2013}. For two facilities, the only strategyproof mechanism that exist places the facility at the leftmost and the rightmost locations (\textsc{Two Extremes} mechanism) and has an approximation ratio at most $n-2$ \cite{Procaccia2013,Fotakis2014}. However, for three or more facilities, there is no deterministic anonymous strategyproof mechanisms for k-Facility Location with a bounded approximation ratio \cite{Fotakis2014}. On the positive side, randomized mechanism achieve better approximation ratios. For 2-Facility Location games \textsc{Proportional Mechanism} achieves a constant approximation ratio of 4 \cite{Lu2010}. The \textsc{Inversely Proportional Mechanism} \cite{escoffier2011} has $n/2$-approximation ratio for $k$-facility location games with $n=k+1$ agents. Fotakis and Tzamos \cite{Fotakis2013sp} proposed the \textsc{Equal Cost} a randomized strategyproof mechanism with an approximation ration at most $n$ for any number of agents.

\begin{table}[ht]
    \centering
    \begin{tabular}{|c|c|c|c|}
         \hline
         & $k=1$ & $k=2$ & $k\ge3$ \\ \hline 
        Deterministic & 1\cite{Moulin1980} & $n-2$ \cite{Procaccia2013} & $\infty$ \cite{Fotakis2013}\\ \hline
        Randomized & 1\cite{Moulin1980} & 4 \cite{Lu2010} & $n$ \cite{Fotakis2013sp} \\ \hline
    \end{tabular}
    \caption{Best known results of approximation ratio for $k$-Facility Location on the line}
    \label{tab:summaryLine}
\end{table}


When the agents are located on more general metric spaces than the line, the problem becomes much more complicated. For tree metrics, for one facility the mechanism that places the facility at the median location is strategyproof and optimal \cite{Schummer2002}, but for two or more facilities there is no deterministic strategyproof mechanism with a bounded approximation. When the agents are located in a circle then any strategyproof mechanism is a dictatorship.

%+++++ extra power to the mechanism imposing mechanism and verification.

We can overcome the difficulty of the problem by giving ``additional power" to the mechanisms. Nissim, Smorodinsky, and Tennenholtz first proposed the class of \emph{imposing mechanisms} \cite{Nissim2010}. In that class, the mechanism can restrict the way agents are allowed to exploit the outcome. In the Facility Location setting, for example, agents are forced to connect to the facility closest to their reported location, even if there is a facility closer to their true location. This way, an agent that misreports has a greater connection cost since she cannot select the facility closest to her ideal location. Using the same techniques, Fotakis and Tzamos \cite{Fotakis2013} showed that the Winner Imposing version of the Proportional mechanism for the $k$-Facility Location is strategyproof and has an approximation ratio of at most $4k$. The mechanism requires that any agent that has a facility at her reported location connect to it. 

Another way to ensure strategyproofness and better approximation ratios is to consider \emph{mechanisms with local verification}. This was first introduced by Green and Laffont \cite{Green1986}. In order to model partial verification, they restricted the set of each agent’s allowable deviations to a so-called correspondence set. Later, Nisan \cite{Nisan2001} proposed a mechanism with verification for scheduling problems in unrelated machines, where each machine is controlled as a selfish agent. The mechanism has two stages: first, in the declaration phase, the agents report their types(namely the time of execution for each job) and the mechanism returns an allocation; then, in the execution phase, each machine executes the tasks it is allocated. The payments are given after the execution, so the mechanism can verify if the agents reported their true types and punish those who lied by reducing their payments. This notion of local verification has been adapted for different settings, see for example \cite{Auletta2009,Carroll2012,Caragiannis2012,Archer2014,Fotakis2015,Fotakis2016}.


Note that in the class of imposing mechanisms, the mechanism does not know whether or not an agent lied, but the way it restricts any agent's post-action options ensures that any lying agent will be penalized. On the other hand, a mechanism with verification can identify agents that lied and punish them. 


%We underline that liars still get utility from the selected outcome. It just happens that their preferences are not taken into account in the allocation. For these reasons, the penalty of exclusion from the mechanism is mild and compatible with the spirit of mechanisms without money

\section{Beyond Worst Case Analysis}

The previous results are based on the worst-case analysis framework in which an algorithm is characterized by its performance on the worst possible input. A good worst-case guarantee shows that an algorithm works well without any assumptions. However, in many interesting problems, this is impossible. To get better insight about an algorithms performance we want to move beyond the worst-case \cite{Rough20}. 

Take the problem of linear programming, where the goal is to minimize (or maximize) a linear objective function subject to linear constraints. Two algorithms solve this problem: the simplex method (exponential time in the worst-case) and the ellipsoid method (polynomial in the worst-case). Yet, empirically, simplex performs way better than the ellipsoid method. 

In this work we are going to focus on $\emph{Perturbation Stability}$. The notion of perturbation stability was first introduced by Bilu and Linial \cite{Bilu2009} for the Max-Cut Problem. Intuitively stability implies that small perturbation on the input does not change the optimal solution. By restricting our attention to stable instances, that model ``real world" instances, we can find the optimal solution. This was later adapted for clustering problems \cite{Balcan2011, Awasthi2012, Balcan2015, Angelidakis2017,Agarwal2020}. Clustering is the task of grouping a set of points in such a way that objects in the same group are more similar (in some sense) to each other than to those in other groups. Clustering, as an optimization problem, is $NP$-hard for most commonly used objective functions (e.g, $k$-median\cite{Mahajan2012}, $k$-center). However, simple clustering algorithms perform well in practice because the clusters are well defined. The notion of stability captures the structure that is usually found in practical instances. Therefore, by restricting our domain to a subset of instances, namely perturbation stable instances, we can design exact algorithms for an $NP$-hard problem.


%{-θέλει αλλαγή-}
We will call a clustering instance $\g$-\emph{perturbation stable} if the optimal clustering remains the same even if we scale down any subset of the entries of the distance matrix by a factor of at most $\g$. By focusing on the geometry of the instance, we show that in stable instances, all the inter-cluster distances are bounded by some intra-cluster distance. We will show three properties that hold in any $\g$-stable instance: \emph{Center-Proximity} (Definition \ref{Cprox}), \emph{Weak Center-Proximity} (Lemma \ref{WCprox}) and \emph{Cluster Separation Property} (Lemma \ref{CSprop}). 

\section{Facility Location on Stable instances and Contribution}



Since clustering and Facility Location games are closely related problems, an interesting question is whether we can achieve better results using the ideas and techniques from perturbation stability. The main difficulty in both problems is identifying the optimal clusters, namely the groups of points or agents that are served by the same center or facility. However, in Facility Location games, the agents are strategic, which means even if only one agent misreports her ideal location, the new instance can have a very different clustering structure than the original. Therefore, the question actually is how much power an agent has in manipulating the output of a mechanism assuming stability of the instance and, consequently, well-defined clusters.





 Fotakis and Patsilinakos \cite{Fotakis2021} initiate the study in that direction. The main idea is that the class of instances that appear in the real world are stable, and therefore, the mechanism should perform well only in those instances. Therefore, they consider the $k$-Facility Location games on the line restricted on perturbation stable instance. They showed that the optimal solution is strategyproof for $(2+\sqrt{3})$-stable instance, if the optimal clustering does not include  any singleton clusters. Let us point out that without any assumptions about the input, the optimal solution is not strategyproof, even for 2-Facility location games on the line. To avoid the restriction on stable instances with out singleton clusters the also proposed a randomized strategyproof mechanism with a constant approximation ratio for $5$-stable instances. Moreover, focusing on stable instances does not make the problem trivial. Extending the impossibility result for $k$-Facility Location games with $k\ge3$, they also proved that there is no deterministic anonymous strategyproof mechanism for $k$-Facility Location, with $k\ge3$, on $(2-\delta)$-stable instances with bounded approximation ratio for any $\delta>0$.

Our goal in this theses is to see how we can extend the previous results into more general metric spaces than the line. In the unrestricted domain the problem becomes non-trivial even when we want to place one in a circle or two facilities in a tree. We show that the optimal solution is strategyproof for $k$-Facility Location for $2+\sqrt{3}$ in tree metrics.



