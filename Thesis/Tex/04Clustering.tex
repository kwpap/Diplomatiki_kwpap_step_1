\section{Motivation}
\begin{enumerate}
    \item Παίζοντας με τα δεδομένα, έγιεν σαφές πως οι δωρεάν άδειες δεν δίνονταν με ομοιογενή τρόπο μεταξύ των χωρών.
    \item Το σύστημα προσπαθεί να είναι δίκαιο, επομένως, είναι σημαντικό να φανεί αν χώρες που μοιάζουν αντιμετωπίζονται με τον ίδιο τρόπο. 
    \item Αποδοτικότητα στο σύστημα;
    \item Συνέχεια της διπλωματικής του παναγιώτη που χώριζε τις εταιρείες σε αρχηγούς και ακόλουθους. 
\end{enumerate}

\subsection{Experiment 1 - Same-same different-different}
\subsection{Definition}
Hypothesis (H1): Countries with similar economic and energy profiles should receive similar levels of free allowances, while countries with distinct profiles receive proportionally different free allowances. 

Null Hypothesis (H0): Countries with similar, economic and energy profiles might receive a differentiated free allocation.  
\subsection{Experiment Design}
Identify Variables:

Independent Variables: Economic and energy indicators that define country profiles (e.g., population, GDP per capita, energy intensity, sector composition).
Dependent Variable: Amount of free allowances allocated to each country under the EU ETS.

Experimental Group: Most EU member countries subject to ETS during the study period (2005–2020). In particular: Austria, Belgium, Bulgaria, Cyprus, Denmark, Estonia, Finland, France, Germany, Greece, Hungary, Italy, Latvia, Lithuania, Luxembourg, Malta, Netherlands, Polands, Portugal, Romania, Slovenia, Spain, Sweden.

Methodology 1:
Calculate the euclidian Distance of the 




\subsection{Data Collection}
Data Sources:
Collect historical data from 2005 to 2020 on population, GDP per capita, inflation, sector composition, energy supply, energy intensity, verified emissions, and free allocated emissions for each EU member state.
Standardize and Normalize Data:
Normalize indicators (e.g., by dividing each value by the overall average for that indicator) to make profiles comparable across countries.
\subsection{Experiment Execution}

\subsection{Conclutions}

Clearly state why clustering is beneficial to understanding the EU ETS. Highlight how clustering helps reveal inefficiencies or disparities.
\section{tools used}
k-means, k finding ideas etc.
\section{Results}
section 2 cest23
\cite{dimosfair}